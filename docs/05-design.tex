\chapter{Конструкторский раздел}

\section{Описание системы автоматической проверки отчета}

\includeimage
{UseCase} % Имя файла без расширения (файл должен быть расположен в директории inc/img/)
{f} % Обтекание (без обтекания)
{h} % Положение рисунка (см. figure из пакета float)
{1\textwidth} % Ширина рисунка
{Диаграмма вариантов автоматической проверки отчета} % Подпись рисунка

С помощью использования алгоритма автоматической проверки отчета возможно существенно сократить временные ресурсы, выделяемые нормоконтроллером на проверку огромного количества отчетов, однако, полностью отказаться от финального контроля результатов человеком невозможно, таким образом существует две роли при проверки отчета на соответствие ГОСТ, а именно: студент и нормоконтроллер.

Студент отправляет отчет на проверку, а затем получает результат со списком ошибок (если имеются). Нормоконтроллер же анализирует отчет, составленный автоматической системой проверки, и при необходимости может внести необходимые правки.
