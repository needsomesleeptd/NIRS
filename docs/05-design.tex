\chapter{Конструкторский раздел}

\section{Описание системы автоматической проверки отчета}
С помощью использования алгоритма автоматической проверки отчета возможно существенно сократить временные ресурсы, выделяемые нормоконтролером на проверку огромного количества отчетов, однако, полностью отказаться от финального контроля результатов человеком невозможно, таким образом существует две роли при проверки отчета на соответствие ГОСТ, а именно студент и нормоконтролер.

Студент отправляет отчет на проверку, а затем получает результат со списком ошибок (если имеются). Нормоконтролер же анализирует отчет, составленный автоматической системой проверки, и при необходимости может внести необходимые поправки
