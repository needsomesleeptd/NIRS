\chapter{Конструкторский раздел}
В неавтоматизированной системе проверки отчетов на соответствие ГОСТ и дополнительным требованиям присутствуют две роли: студент, выполняющий некоторую работы, которая подразумевает написание отчета и нормоконтроллер, принимающий экспертное решение о соответствии предоставленного ему отчета необходимым требованиям.

Использование автоматической проверки отчетов на соответствие ГОСТ и дополнительным требованиям сократит временные затраты на проверку отчетов.

\section{Описание системы автоматической проверки отчета на соответствие ГОСТ и дополнительным требованиям}

\includeimage
{UseCase} % Имя файла без расширения (файл должен быть расположен в директории inc/img/)
{f} % Обтекание (без обтекания)
{H} % Положение рисунка (см. figure из пакета float)
{0.9\textwidth} % Ширина рисунка
{Диаграмма вариантов автоматической проверки отчета} % Подпись рисунка

\includeimage
{ka} % Имя файла без расширения (файл должен быть расположен в директории inc/img/)
{f} % Обтекание (без обтекания)
{H} % Положение рисунка (см. figure из пакета float)
{1\textwidth} % Ширина рисунка
{Диаграмма состояний проверки отчета} % Подпись рисунка

С помощью использования автоматической проверки отчета возможно сократить временные ресурсы, выделяемые нормоконтроллером на проверку отчетов студентов, однако, полностью отказаться от финального контроля результатов человеком невозможно, таким образом существует две роли при проверки отчета на соответствие ГОСТ, а именно: студент и нормоконтроллер.

\includeimage
{stepsDiagram} % Имя файла без расширения (файл должен быть расположен в директории inc/img/)
{f} % Обтекание (без обтекания)
{H} % Положение рисунка (см. figure из пакета float)
{0.8\textwidth} % Ширина рисунка
{Диаграмма последовательности действий} % Подпись рисунка

Студент отправляет отчет на проверку, а затем получает результат со списком ошибок (если имеются). Нормоконтроллер же анализирует отчет, составленный автоматической системой проверки, и при необходимости может внести необходимые правки.

Использование автоматической проверки отчетов на соответствие ГОСТ позволяет эффективнее использовать временные ресурсы нормоконтроллера.

\section*{Вывод}
В данном разделе была описана система автоматической проверки отчетов студентов на соответствие ГОСТ и дополнительным требованиям.
