\chapter{Аналитический раздел}
\section{Виды PDF форматов}
В данной части работы будут проанализированы существующие виды PDF документов.
Существует несколько различных видов PDF документов, каждый из которых имеет свои особенности и ограничения:
\begin{enumerate}
	\item PDF/A;
	\item PDF/X;
	\item PDF/E;
	\item PDF/UA.
\end{enumerate}
Рассмотрим каждый из них по отдельности.
\subsection{PDF/A}
Данный формат, предназначенный для долгосрочного хранения документов. Он обеспечивает сохранность и неприкосновенность содержимого даже через длительные периоды времени. Однако, PDF/A ограничен в функциональности и не поддерживает некоторые расширенные возможности форматов PDF.
Данный формат также разделяется на несколько подклассов: PDF/A-1, PDF/A-2, PDF/A-3, PDF/A-4.
\subsubsection{PDF/A-1}
PDF/A-1 - самый распространенный формат оригинального PDF/A на сегодняшний день. Он основан на PDF 1.4 и является наиболее ограниченным, так как не поддерживает JPEG 2000, вложения, слои и прозрачность. Часть 1 стандарта была опубликована 28 сентября 2005 года и определяет два уровня соответствия для файлов PDF:
\begin{enumerate}

\item PDF/A-1b - уровень B (базовое) соответствие.
Уровень B соответствия требует только те стандарты, которые необходимы для надежного воспроизведения внешнего вида документа. Это означает, что файл будет выглядеть так же при просмотре и/или печати в ближайшем или дальнем будущем \cite{pdf_a_versions};

\item PDF/A-1a - уровень A (доступное) соответствие.

Уровень A соответствия включает все требования уровня B, а также функции, предназначенные для улучшения доступности документа путем добавления атрибутов, обеспечивающих доступность документов для пользователей с ограниченными возможностями. Для сохранения логической структуры и естественного порядка чтения документа требуется надежная семантическая и структурная информация в формате Unicode \cite{pdf_a_versions}.
\end{enumerate}

\subsubsection{PDF/A-2}
PDF/A-2 представляет собой ряд новых функций:
\begin{enumerate}
\item Сжатие JPEG2000 было введено с спецификацией PDF 1.5, которая была выпущена после времени выпуска стандарта PDF/A-1. Добавление сжатия JPEG2000 особенно полезно для отсканированных документов, таких как карты, книги, а также документов с цветным содержимым, таких как чеки или паспорта;
\item Вложенные файлы PDF/A через коллекции: Acrobat позволяет пользователям создавать коллекции (иногда также называемые "портфелями"), где несколько документов PDF/A объединяются в один "контейнерный" документ PDF. Одним из возможных применений коллекции PDF/A является архивирование электронных писем, где вложения электронной почты могут быть преобразованы в PDF/A и сохранены в виде "коллекций" внутри преобразованного электронного письма PDF/A. Коллекции PDF/A также могут быть полезными для приложений безопасности, где подпись может быть применена к отдельным страницам. Коллекция PDF/A затем объединяет подписанные отдельные страницы. Отдельные страницы могут быть удалены без нарушения действительности подписей оставшихся страниц;
\item Прозрачность: Хотя прозрачность является частью PDF 1.4, на момент выпуска стандарта PDF/A-1 она не была достаточно хорошо определена, чтобы быть включенной в стандарт PDF/A-1. Спецификация существенно совершенствовалась с тех пор, и прозрачность стала общей характеристикой документов PDF. Прозрачность часто используется в виде теней, плавных переходов и выделений.
Необязательное содержимое (слои): Необязательное содержимое, иногда также называемое слоями, полезно для приложений картографии или инженерных чертежей, где отдельные слои могут быть показаны или скрыты в соответствии с требованиями просмотра. Еще одна область применения - руководства пользователя по продуктам, которые продается международно, где разные языки могут быть реализованы на разных слоях;
\item Новый уровень соответствия PDF/A-2u - "u" для Unicode: PDF/A-1b и PDF/A-2b сосредоточены на визуальной целостности, где "b" означает "базовый". PDF/A-1a и PDF/A-2a сосредоточены на доступности - отсюда и обозначение "a". Новым для PDF/A-2 является уровень соответствия PDF/A-2u ("u" для "Unicode"). Он упрощает поиск и копирование текста Unicode для цифровых PDF-документов и PDF-документов, которые были отсканированы с последующим оптическим распознаванием символов (OCR);
\item Метаданные на уровне объекта XMP: PDF/A-2 определяет требования к настраиваемым метаданным XMP;
\item Типы комментариев и аннотаций: Некоторые из новых типов комментариев были добавлены в список запрещенных типов аннотаций, и в то же время некоторые из новых типов комментариев, такие как комментарии редактирования текста, теперь допустимы для стандарта PDF/A-2;
\item Цифровые подписи: В то время как PDF/A-1 уже позволяет использовать цифровые подписи, PDF/A-2 определяет правила, которые должны быть применены для гарантии взаимодействия \cite{pdf_a_2}.
\end{enumerate}
\subsubsection{PDF/A-3}
PDF/A-3 полностью аналогичен PDF/A-2, однако поддерживает добавление любых файлов, а не только PDF типа A. Однако не гарантирует валидность их прочтения в будущем.

Также стоит отметить, что файлы данного вида возможно использовать в электронном документообороте \cite{nalogi}.
\subsubsection{PDF/A-4}
Основное отличие данного вида, является замена уровней соответствия b и u с целью упростить стандарт. PDF/A-4 требует отображения в Юникоде для всех шрифтов в любое время \cite{pdf_a_4}.

\subsection{PDF/X}
Формат, разработанный специально для обмена и печати документов в издательской отрасли. Он обеспечивает точность цветов и расположения элементов страницы, что особенно важно при печати. Однако, PDF/X имеет ограниченные возможности вставки мультимедийных элементов и интерактивности \cite{abdobe_PDF}.


\subsection{PDF/E}
Формат, предназначенный для обмена и хранения документов в инженерной отрасли. Он поддерживает вставку трехмерных моделей, векторных изображений и других инженерных элементов. Однако, PDF/E может быть ограничен в возможности обработки сложных макетов и мультимедийных элементов \cite{abdobe_PDF}.
\subsection{PDF/UA}
Формат, предназначенный для создания доступных документов для пользователей с ограниченными возможностями. Он обеспечивает структурированное представление контента и поддержку технологий чтения вслух и управления навигацией. Однако, PDF/UA может иметь ограничения в отображении сложных макетов и интерактивных элементов \cite{abdobe_PDF}.



