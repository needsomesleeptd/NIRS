\chapter{Исследовательский раздел}
В данном разделе будут рассмотрены методы классификации и проверки выбранных объектов документа на валидность.

\section{Анализ изображений}
Для анализа изображений (таблиц, схем, списка информационных ресурсов) необходимо получить их представление из отчета. Так как изображения могут быть представлены
в векторном формате, то необходимо решать задачу детекции изображений. 	



\subsection{Использование соответствия по шаблону}
Предположение: наличие отличительных объектов на картинке, не встречающийся в других (например, оси для графиков) позволит классифицировать объект.
Для поиска объектов используется поиск по шаблону~\cite{pattern_matching}.

Однако объект, представленный на изображении может находиться в любом положении и под любым наклоном, таким образом для поиска соответствия необходимо 
рассматривать все возможные повороты шаблона. Например при необходимости поиска изображения стрелки  в изображении \ref{img:arrowDetection} с использованием шаблона \ref{img:template}. При использовании метода \textbf{CV-TM-CCOEFF-NORMED} — корреляция Пирсона, результаты представлены на изображении \ref{img:exp-template}, перед использованием шаблона изображение было переведено в  вид оттенков серого(один канал).
Было найдено только одно изображение стрелки, без учета ее поворота, что не позволяет использовать данный метод при всевозможных ее поворотах.

\includeimage
{arrowDetection} % Имя файла без расширения (файл должен быть расположен в директории inc/img/)
{f} % Обтекание (без обтекания)
{h} % Положение рисунка (см. figure из пакета float)
{1\textwidth} % Ширина рисунка
{Пример изображения для поиска шаблона} % Подпись рисунка


\includeimage
{template} % Имя файла без расширения (файл должен быть расположен в директории inc/img/)
{f} % Обтекание (без обтекания)
{h} % Положение рисунка (см. figure из пакета float)
{1\textwidth} % Ширина рисунка
{Шаблон изображения для поиска} % Подпись рисунка



\includeimage
{exp-template} % Имя файла без расширения (файл должен быть расположен в директории inc/img/)
{f} % Обтекание (без обтекания)
{h} % Положение рисунка (см. figure из пакета float)
{1\textwidth} % Ширина рисунка
{Результаты применения шаблона} % Подпись рисунка



\subsection{Использование YOLOv8 для детекциии изображений}
Для детекции изображений была использована модель YOLOv8 \cite{YOLOv8}. Для разметки изображений был использован labelimg \cite{labelimg}. Данные для разметки были взяты из отчетов студентов по предмету Анализ Алгоритмов.
Было выделено 5 классов изображений:
\begin{enumerate}
	\item формулы(имеют метку eq);
	\item схема(имеют метку scheme);
	\item таблица(имеют метку table);
	\item график(имеют метку graph);
	\item список информационных ресурсов(имеют метку lit);
\end{enumerate}

После чего на 294 изображениях была обучена модель YOLOv8, с гиперпараметрами обучения $iou=0.5$,$conf=0.001$ на 10 эпохах.
Результаты полученных изображений приведены на рисунке \ref{img:yolov8-results}.


\includeimage
{yolov8-results} % Имя файла без расширения (файл должен быть расположен в директории inc/img/)
{f} % Обтекание (без обтекания)
{h} % Положение рисунка (см. figure из пакета float)
{1\textwidth} % Ширина рисунка
{Результаты использования модели} % Подпись рисунка





