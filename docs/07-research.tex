\chapter{Исследовательский раздел}
В данном разделе будут рассмотрены методы классификации и проверки выбранных объектов документа на валидность.

\section{Анализ изображений}
Для анализа изображений (таблиц, схем, списка информационных ресурсов) необходимо получить их представление из отчета. Так как изображения могут быть представлены
в векторном формате, то необходимо решать задачу детекции изображений. 	



\subsection{Использование соответствия по шаблону}
Предположение: наличие отличительных объектов на картинке, не встречающийся в других (например, оси для графиков) позволит классифицировать объект.
Для поиска объектов используется поиск по шаблону~\cite{pattern_matching}.

Однако объект, представленный на изображении может находиться в любом положении и под любым наклоном, таким образом для поиска соответствия необходимо 
рассматривать все возможные повороты шаблона. Например при необходимости поиска изображения стрелки  в изображении \ref{img:arrowDetection} с использованием шаблона \ref{img:template}. При использовании метода \textbf{CV-TM-CCOEFF-NORMED} — корреляция Пирсона, результаты представлены на изображении \ref{img:exp-template}, перед использованием шаблона изображение было переведено в  вид оттенков серого(один канал).
Было найдено только одно изображение стрелки, без учета ее поворота, что не позволяет использовать данный метод при всевозможных ее поворотах.

\includeimage
{arrowDetection} % Имя файла без расширения (файл должен быть расположен в директории inc/img/)
{f} % Обтекание (без обтекания)
{H} % Положение рисунка (см. figure из пакета float)
{1\textwidth} % Ширина рисунка
{Пример изображения для поиска шаблона} % Подпись рисунка


\includeimage
{template} % Имя файла без расширения (файл должен быть расположен в директории inc/img/)
{f} % Обтекание (без обтекания)
{H} % Положение рисунка (см. figure из пакета float)
{1\textwidth} % Ширина рисунка
{Шаблон изображения для поиска} % Подпись рисунка



\includeimage
{exp-template} % Имя файла без расширения (файл должен быть расположен в директории inc/img/)
{f} % Обтекание (без обтекания)
{H} % Положение рисунка (см. figure из пакета float)
{1\textwidth} % Ширина рисунка
{Результаты применения шаблона} % Подпись рисунка



\subsection{Использование YOLOv8 для детекциии изображений}
Для детекции изображений была использована модель YOLOv8 \cite{YOLOv8}. Для разметки изображений был использован labelimg \cite{labelimg}. Данные для разметки были взяты из отчетов студентов по предмету Анализ Алгоритмов.
Было выделено 5 классов изображений:
\begin{enumerate}
	\item формулы(имеют метку eq);
	\item схема(имеют метку scheme);
	\item таблица(имеют метку table);
	\item график(имеют метку graph);
	\item список информационных ресурсов(имеют метку lit);
\end{enumerate}

\subsubsection{Обучение на 10 эпохах}
На 267 изображениях была обучена модель YOLOv8, с гиперпараметрами обучения $iou=0.5$,$conf=0.001$ на 10 эпохах, 30 изображений было выделено в валидационную выборку.
Результаты полученных изображений приведены на рисунке~\ref{img:yolov8-results}.
На рисунках \ref{img:nirs-recall}--\ref{img:nirs-mAP50}, представлены метрики после обучения на 10 эпохах, 
после каждой эпохи метрики вычислялись на валидационной выборке,
число 50 после $mAP$  означает порог $iou$ в 50 процентов.
Значения метрик  на валидационной выборке из 30 изображений при 10 эпохах обучения:
\begin{enumerate}
	\item $precision=0.217$;
	\item $recall=0.216$;
	\item $mAP=0.284$.
\end{enumerate}
Результаты работы данной модели на валидационной выборке представлены на картинке \ref{img:yolov8-pred-10}.


\includeimage
{yolov8-results} % Имя файла без расширения (файл должен быть расположен в директории inc/img/)
{f} % Обтекание (без обтекания)
{H} % Положение рисунка (см. figure из пакета float)
{1\textwidth} % Ширина рисунка
{Результаты использования модели при обучении на 10 эпохах} % Подпись рисунка

\includeimage
{nirs-recall} % Имя файла без расширения (файл должен быть расположен в директории inc/img/)
{f} % Обтекание (без обтекания)
{H} % Положение рисунка (см. figure из пакета float)
{1\textwidth} % Ширина рисунка
{Значение метрики recall при обучении на 10 эпохах} % Подпись рисунка


\includeimage
{nirs-precision} % Имя файла без расширения (файл должен быть расположен в директории inc/img/)
{f} % Обтекание (без обтекания)
{H} % Положение рисунка (см. figure из пакета float)
{1\textwidth} % Ширина рисунка
{Значение метрики precision при обучении на 10 эпохах} % Подпись рисунка


\includeimage
{nirs-mAP50} % Имя файла без расширения (файл должен быть расположен в директории inc/img/)
{f} % Обтекание (без обтекания)
{H} % Положение рисунка (см. figure из пакета float)
{1\textwidth} % Ширина рисунка
{Значение метрики mAP50 при обучении на 10 эпохах} % Подпись рисунка

\subsubsection{Обучение на 100 эпохах}
\label{res:learn-100}
Также была поптыка обучения на 100 эпохах без изменения гиперпараметров, однако обучение было остановлено на 73 эпохе оптимизатором yolov8, так как предсказания модели не улучшились
за последние 50 эпох, наилучшие результаты предсказаний были получены на 23 эпохе. Результаты работы данной модели на валидационной выборке представлены на картинке \ref{img:yolov8-pred-100}.
Результаты полученных изображений приведены на рисунке~\ref{img:yolov8-results-100}.
На рисунках \ref{img:nirs-recall-100}--\ref{img:nirs-mAP50-100}, представлены метрики после обучения на 73 эпохах.
Значения метрик  на валидационной выборке из 30 изображений при 23 эпохах обучения:
\begin{enumerate}
	\item $precision=0.364$;
	\item $recall=0.9603$;
	\item $mAP=0.74$.
\end{enumerate}


\includeimage
{yolov8-results-100} % Имя файла без расширения (файл должен быть расположен в директории inc/img/)
{f} % Обтекание (без обтекания)
{H} % Положение рисунка (см. figure из пакета float)
{1\textwidth} % Ширина рисунка
{Результаты использования модели после обучения на 73 эпохах} % Подпись рисунка

\includeimage
{nirs-recall-100} % Имя файла без расширения (файл должен быть расположен в директории inc/img/)
{f} % Обтекание (без обтекания)
{H} % Положение рисунка (см. figure из пакета float)
{1\textwidth} % Ширина рисунка
{Значение метрики recall при обучении на 73 эпохах} % Подпись рисунка


\includeimage
{nirs-precision-100} % Имя файла без расширения (файл должен быть расположен в директории inc/img/)
{f} % Обтекание (без обтекания)
{H} % Положение рисунка (см. figure из пакета float)
{1\textwidth} % Ширина рисунка
{Значение метрики precision при обучении на 73 эпохах} % Подпись рисунка


\includeimage
{nirs-mAP50-100} % Имя файла без расширения (файл должен быть расположен в директории inc/img/)
{f} % Обтекание (без обтекания)
{H} % Положение рисунка (см. figure из пакета float)
{1\textwidth} % Ширина рисунка
{Значение метрики mAP50 при обучении на 73 эпохах} % Подпись рисунка

Значение метрики $precision$  меньше других значений рассматриваемых метрик, модель чаще ошибается при классификации объектов, чем при их выделении (метрика $mAP$).
Для получения более точных результатов  классификации необходимо сбалансировать классы (рассматривать  одинаковое количество объектов каждого класса) и
увеличить их количество.




