\chapter{Технологический раздел}

\section{Средства реализации}


\subsection{Используемые библиотеки}

\subsubsection{OpenCV}

OpenCV (Библиотека компьютерного зрения с открытым исходным кодом) - это библиотека программного обеспечения для компьютерного зрения и машинного обучения с открытым исходным кодом.

Библиотека содержит более 2500 оптимизированных алгоритмов, которые включают в себя полный набор как классических, так и самых современных алгоритмов компьютерного зрения и машинного обучения. Эти алгоритмы могут быть использованы для обнаружения и распознавания лиц, идентификации объектов, классификации действий человека в видео, отслеживания движений камеры, отслеживания движущихся объектов, поиска похожих изображений из база данных изображений,  распознавание пейзажа и т.~д.~\cite{about_openCV}.

Данная библиотека реализуют следующий функционал:
\begin{enumerate}
	\item поиск по шаблону (англ. template matching), данная функция позволяет 
	находить на изображении большего размера шаблон меньшего размера и выделять его, данная функция упростит поиск геометрических примитивов на изображении  ~\cite{pattern_matching};
	\item классификация  изображений из модуля глубоких нейронных сетей (англ. dense neural networks module) позволит разбивать изображения на необходимые подклассы~\cite{DL_openCV}.
	
	\item Благодаря оптическому распознаванию текста (англ. optical character recognition) возможно определение местоположения и получение информации о содержании текста~\cite{OCR_openCV}.
\end{enumerate}


\subsection{YOLO}

Популярная модель обнаружения объектов и сегментации изображений YOLO (You Only Look Once) была разработана Джозефом Редмоном и Али Фархади из Вашингтонского университета. YOLOv8, использующаяся в данной работе является эволюцией серий моделей YOLO \cite{YOLOv8}.
\begin{enumerate}
\item Модель YOLOv2, выпущенная в 2016 г., была усовершенствована за счет использования пакетной нормализации, якорных блоков и размерных кластеров.
\item YOLOv3, выпущенная в 2018 году, позволила еще больше повысить производительность модели за счет использования более эффективной опорной сети, множества якорей и объединения пространственных пирамид.
В 2020 году была выпущена YOLOv4, в которой появились такие инновации, как увеличение данных Mosaic, новая головка обнаружения без якорей и новая функция потерь.
\item YOLOv5 позволила еще больше повысить производительность модели и добавить такие новые возможности, как оптимизация гиперпараметров, интегрированное отслеживание экспериментов и автоматический экспорт в популярные экспортные форматы.
\item YOLOv6 была открыта компанией Meituan в 2022 году и используется во многих автономных роботах-доставщиках компании.
\item В YOLOv7 добавлены дополнительные задачи, такие как оценка позы по набору данных COCO keypoints.
\item YOLOv8 - это последняя версия YOLO от Ultralytics. Являясь передовой, современной (SOTA) моделью, YOLOv8 опирается на успех предыдущих версий, представляя новые возможности и улучшения для повышения производительности, гибкости и эффективности. YOLOv8 поддерживает полный спектр задач искусственного интеллекта, включая обнаружение, сегментацию, оценку положения, отслеживание и классификацию. Такая универсальность позволяет пользователям использовать возможности YOLOv8 в различных приложениях и областях.
\end{enumerate}







