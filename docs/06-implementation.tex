\chapter{Технологический раздел}

\section{Средства реализации}


\subsection{Используемые библиотеки}

\subsubsection{OpenCV}

OpenCV (Библиотека компьютерного зрения с открытым исходным кодом) - это библиотека программного обеспечения для компьютерного зрения и машинного обучения с открытым исходным кодом.

Библиотека содержит более 2500 оптимизированных алгоритмов, которые включают в себя полный набор как классических, так и самых современных алгоритмов компьютерного зрения и машинного обучения. Эти алгоритмы могут быть использованы для обнаружения и распознавания лиц, идентификации объектов, классификации действий человека в видео, отслеживания движений камеры, отслеживания движущихся объектов, поиска похожих изображений из база данных изображений,  распознавание пейзажа и т.~д.~\cite{about_openCV}.

Данная библиотека реализуют следующий функционал:
\begin{enumerate}
	\item поиск по шаблону (англ. template matching), данная функция позволяет 
	находить на изображении большего размера шаблон меньшего размера и выделять его, данная функция упростит поиск геометрических примитивов на изображении  ~\cite{pattern_matching};
	\item классификация  изображений из модуля глубоких нейронных сетей (англ. dense neural networks module) позволит разбивать изображения на необходимые подклассы~\cite{DL_openCV}.
	
	\item Благодаря оптическому распознаванию текста (англ. optical character recognition) возможно определение местоположения и получение информации о содержании текста~\cite{OCR_openCV}.
\end{enumerate}


\subsection{Использование поиска по шаблону}




