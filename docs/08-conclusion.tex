\chapter*{ЗАКЛЮЧЕНИЕ}
\addcontentsline{toc}{chapter}{ЗАКЛЮЧЕНИЕ}

В ходе выполнения работы были выполнены следующие задачи:
\begin{itemize}
	\item проанализированы существующие виды PDF-документов и связанные с ними ограничения;
	\item классифицированы типовые требования и ошибки при оформлении отчётов: текста, рисунков, графиков, схем алгоритмов, таблиц и списка источников;
	\item проанализированы существующие решения выделения составных частей (элементов) отчёта, представленного в формате PDF, в соответствии с ГОСТ 7.32 (фрагменты текста, рисунки, графики, схемы алгоритмов, источники, таблицы и пр.) для дальнейшего анализа с использованием средств компьютерного зрения и автоматического анализа текста;
	\item реализовано программное обеспечение, позволяющее выделить составные части (элементы) отчета, представленного в формате PDF для дальнейшего анализа на соответствие ГОСТ, работа которого возможна на операционной системе <<Astra Linux>>.
\end{itemize}

В ходе обучение модели YOLOv8 было использовано 297 изображений (267 изображений для обучения и 30 для проверки корректности работы). Для оценки качества работы модели были выбраны следующие метрики:
\begin{itemize}
\item precision;
\item recall;
\item mAP.
\end{itemize}
Наилучшие результаты были получены при обучении на 23 эпохах, при наблюдении метрик \ref{res:learn-100}, значение метрики $precision = 0.364$, меньше значений
других метрик, можно сделать вывод, что для более точной детекции изображений необходимо увеличить размер валидационной и тренировочной выборок.

Следующим шагом в разработке автоматической проверки отчетов является разработка отдельных алгоритмов проверки на соответствие ГОСТ каждого из выделенных элементов отчета.
