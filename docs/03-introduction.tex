\chapter*{ВВЕДЕНИЕ}
\addcontentsline{toc}{chapter}{ВВЕДЕНИЕ}

Во время обучения студентам регулярно приходится писать отчеты к различным видам работ (курсовые, лабораторные, научно-исследовательские работы и т. д.), при этом оформление работ должно соответствовать ГОСТ, что необходимо своевременно проверить и при необходимости отправить отчет на доработку, однако, количество студентов намного превышает количество нормоконтроллеров. Для ускорения процесса проверки возможно использование автоматических систем.

Для достижения цели научно-исследовательской работы требуется решить следующие задачи:
\begin{itemize}
	\item проанализировать существующие виды PDF-документов и связанных с ними ограничений;
	\item классифицировать типовые требования и ошибки при оформлении отчётов: текста, рисунков, графиков, схем алгоритмов, таблиц, списка источников и т.~д.;
	\item проанализировать существующие решения и разработать алгоритм выделения составных частей (элементов) отчёта, представленного в формате PDF, в соответствии с ГОСТ 7.32 (фрагменты текста, рисунки, графики, схемы алгоритмов, источники, таблицы и пр.) для дальнейшего анализа с использованием средств компьютерного зрения и автоматического анализа текста;
	\item проанализировать существующие решения и разработать алгоритм проверки рисунков на соответствие ГОСТ 7.32 и дополнительным требованиям;
	\item проанализировать существующие решения и разработать алгоритм классификации рисунков по содержанию: графики, схемы алгоритмов, UML-диаграммы, IDEF0, BPMN2.0 и прочие изображения;
	\item проанализировать существующие решения и разработать алгоритм проверки схемы алгоритма на соответствие ГОСТ 7.32 и дополнительным требованиям;
	\item проанализировать существующие решения и разработать алгоритм проверки текста и списка используемых источников на соответствие ГОСТ 7.32 и дополнительным требованиям;
	\item реализовать предложенные алгоритмы в едином ПО для целевой ОС <<Astra Linux>> и <<ROSA Linux>>.
\end{itemize}
