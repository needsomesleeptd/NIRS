\chapter*{ВВЕДЕНИЕ}
\addcontentsline{toc}{chapter}{ВВЕДЕНИЕ}

Во время обучения студентам не раз приходится писать отчеты к различным видам работ (курсовые, лабораторные, научно-исследовательские работы и т.п.), при этом все эти работы должны быть своевременно проверены и оценены, а также, возможно, отправлены на доработку. Однако, количество студентов намного превышает количество нормоконтроллеров, которые оценивают работы, чтобы ускорить процесс оценивания возможно использование автоматических систем проверки, которые могут генерировать отчет, содержащий результаты проверки работы на наличие наиболее распространенных видов ошибок.

Целью данной научно-исследовательской работы является создание прототипа системы автоматической проверки работ студентов.

Для достижения поставленной цели требуется решить следующие задачи:
\begin{itemize}
	\item проанализировать существующие виды PDF-документов и связанных с ними ограничений;
	\item классифицировать типовые требования и ошибки при оформлении отчётов: текста, рисунков, графиков, схем алгоритмов, таблиц, списка источников и т.д.;
	\item проанализировать существующие решения и разработать алгоритм выделения составных частей (элементов) отчёта, представленного в формате PDF, в соответствии с ГОСТ 7.32 (фрагменты текста, рисунки, графики, схемы алгоритмов, источники, таблицы и пр.) для дальнейшего анализа с использованием средств компьютерного зрения и автоматического анализа текста;
	\item проанализировать существующие решения и разработать алгоритм проверки рисунков на соответствие ГОСТ 7.32 и дополнительным требованиям;
	\item проанализировать существующие решения и разработать алгоритм классификации рисунков по содержанию: графики, схемы алгоритмов, UML-диаграммы, IDEF0, BPMN2.0 и прочие изображения;
	\item проанализировать существующие решения и разработать алгоритм проверки схемы алгоритма на соответствие ГОСТ 7.32 и дополнительным требованиям;
	\item проанализировать существующие решения и разработать алгоритм проверки текста и списка используемых источников на соответствие ГОСТ 7.32 и дополнительным требованиям;
	\item реализовать предложенные алгоритмы в едином ПО для целевой ОС <<Astra Linux>> и <<ROSA Linux>>.
\end{itemize}
