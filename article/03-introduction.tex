\chapter*{ВВЕДЕНИЕ}
\addcontentsline{toc}{chapter}{ВВЕДЕНИЕ}

Во время обучения студентам не раз приходится писать отчеты к различным видам работ (курсовые, лабораторные, научно-исследовательские работы и т.п.), при этом все эти работы должны быть своевременно проверены и оценены, а также, возможно, отправлены на доработку. Однако, количество студентов намного превышает количество нормоконтроллеров, которые оценивают работы, чтобы ускорить процесс оценивания возможно использование автоматических систем проверки, которые могут генерировать отчет, содержащий результаты проверки работы на наличие наиболее распространенных видов ошибок. Однако, изучая данную проблему не было обнаружено готовых решений, позволяющих провести автоматический анализ работ студентов.

Первым этапом в разработке системы автоматической проверки отчетов является выделение составных ключевых (информативных) элементов отчета таких как: таблицы, графики, схемы алгоритмов и список литературы с использованием средств компьютерного зрения, для дальнейшего анализа полученных элементов на соответствие ГОСТ.
