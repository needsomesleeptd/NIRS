%\chapter{Аналитический раздел}

\textbf{Общие ошибки в отчетах}. Выход за границы листа является одной из самых распространенных ошибок. В ГОСТ 7.32 указаны следующие размеры полей: левое — 30 мм, правое — 15 мм, верхнее и нижнее — 20 мм. 

Каждый объект (например: таблица, рисунок, схема алгоритма, формула) должен быть подписан и пронумерован, однако более подробно подписи к каждому из них будут рассмотрены в следующих разделах.

\textbf{Ошибки в тексте}. Страницы отчета должны быть пронумерованы, однако, номер на титульном листе не ставится (но он является первой страницей, это означает, что следующая страница должна иметь номер 2).

Ненумерованный заголовок (введение, список литературы, оглавление и т.п.) должен быть выравнен по центру, при этом он состоит только из прописных букв.

Абзацный отступ должен быть одинаковым по всему тексту отчета и равен 1,25 см. Возможна потеря научного стиля и переход к публицистике, что является ошибкой. Также текст работы должен быть написан на государственном языке.

\textbf{Ошибки в рисунках}. Каждый рисунок должен быть подписан, при этом подпись должна располагаться строго по центру, внизу рисунка.

Все рисунки должны быть выполнены в высоком качестве, если обратное не требуется в самой работе.

Если рисунок не вмещается в ширину страницы, то допускается повернуть его таким образом, чтобы верх рисунка был ближе к левой части страницы.

\textbf{Ошибки в графиках}. Для каждого графика должна существовать легенда, для оформления которой есть два варианта: в одном из углов графика находится область, в которой указаны все обозначения или в подписи к графику описано каждое обозначение. Должны быть подписаны единицы измерения каждой из осей.

Отчеты могут быть напечатаны в черно-белом варианте, поэтому на графиках должны быть маркеры, которые позволят отличить графики друг от друга даже не в цветом варианте. Также при большом количестве графиков на одном рисунке возможна ситуация, при которой невозможно отличить один график от другого, что является ошибкой.

\textbf{Ошибки в схемах алгоритмов}. Если схема не влезает на одну страницу, то она разбивается на несколько частей, каждая из них должна быть подписана. Для разделения схемы алгоритма на части используется специальный символ-соединитель, который отображает выход в часть схемы и вход из другой части этой схемы, соответствующие символы-соединители должны содержать одно и то же уникальное обозначение.

Довольно часто вместо символа начала или конца алгоритма используют овал, однако в этом случае должен быть использован прямоугольник с закругленными углами.

Также при использовании символа процесса (прямоугольник) используют прямоугольник с закругленными углами.

При соединении символов схемы алгоритмов не нужны стрелки, если они соединяют символы в направлении слево-направо или сверху-вниз, в остальных случаях символы должны соединяться линиями со стрелкой на конце.

При использовании символа процесса---решение как минимум одна из соединительных линий должна быть подписана, однако возможен также вариант, когда подписаны обе линии.

Пояснительный текст не должен пересекаться с символами, использующимися для составления схем.

\textbf{Ошибки в таблицах}. Каждая таблица должна быть подписана. Наименование следует помещать над таблицей слева, без абзацного отступа в следующем формате: Таблица Номер таблицы - Наименование таблицы. Наименование таблицы приводят с прописной буквы без точки в конце.

Таблицу с большим количеством строк допускается переносить на другую страницу. При переносе части таблицы на другую страницу слово «Таблица», ее номер и наименование указывают один раз слева над первой частью таблицы, а над другими частями также слева пишут слова «Продолжение таблицы» и указывают номер таблицы.

\textbf{Ошибки в формулах}. Каждая формула должна быть пронумерована вне зависимости от того, есть ли на нее ссылка в тексте или нет. Нумерация может осуществляться в двух вариантах: сквозная нумерация (номер формулы не зависит от раздела, в котором она находится) или нумерация, зависящая от раздела (в том случае номер формулы начинается с номера раздела).

После каждой формулы должен находиться знак препинания (точка, запятая и т.п.). Если в формуле содержится система уравнений, то после каждого из них (за исключением последнего) ставится запятая, а после последнего --- точка, либо запятая.

Номер формулы должен быть выравнен по правому краю страницы и находиться по центру формулы.

Если формула вставляется в начале страницы, то часто перед ней может присутствовать отступ, которого быть не должно.

\textbf{Ошибки в списках}. Ненумерованные списки должны начинаться с удлиненного тире, а также после номера пункта обязательно должна стоять скобка.

В конце каждого пункта списка должен быть знак препинания, от которого зависит первая буква первого слова следующего пункта: если пункт заканчивается на точку, то первое слово следующего пункта должно начинаться на прописную букву, если пункт заканчивается запятой или точкой с запятой, то следующий первое слово следующего слово должно начинаться со строчной буквы.

\textbf{Ошибки в списке литературы}. Часто при описании одного из источников не указывается одна из составных частей (автор, издательство и т.п.). Также нередко встречаются ссылки на так называемые «препринтовские» издательства (сама статья еще не вышла).
